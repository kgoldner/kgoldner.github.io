\documentclass[11pt]{article}
\usepackage{fullpage}
\usepackage{algorithm}
\usepackage[noend]{algorithmic}
\usepackage{amsmath,amssymb,amsthm}



% These define new environments / formats for lemmas, definitions, running time, etc.
\newtheorem{lemma}{Lemma}
\newtheorem{definition}{Definition}
\newtheorem{notation}{Notation}
\newtheorem*{claim}{Claim}
\newtheorem{observation}{Observation}
\newtheorem{conjecture}[lemma]{Conjecture}
\newtheorem{theorem}[lemma]{Theorem}
\newtheorem{corollary}[lemma]{Corollary}
\newtheorem{proposition}[lemma]{Proposition}
\newtheorem*{rt}{Running Time}


% These define nice ways to format P and OPT (use \P or \opt)
\def\P{\ensuremath{\mathcal{P}}}
\def\opt{\ensuremath{\textsc{opt}}}


% enumerate uses a., b., c., ...
\renewcommand{\labelenumi}{\bf \alph{enumi}.}


%%% TODO: Fill out the appropriate homework number and your name below
%%%
%%%
% Changes the title box on the first page
\renewcommand\maketitle{
\begin{center}
\begin{tabular*}{6.44in}{l @{\extracolsep{\fill}}c r}
\bfseries  &  & \bfseries DS 320 Spring 2022 \\
\bfseries&  & \bfseries  Homework \#? Solutions  \\
\bfseries   &   &  \bfseries My Name \\ 
\end{tabular*}
\end{center} }




%%
%%
%% THE REAL STUFF STARTS HERE
%%
%%
\begin{document}
\maketitle


%%% TODO: PLEASE LIST YOUR COLLABORATORS HERE
\noindent Collaborators:


\subsection*{Part  1: Big-Oh No!}
% TODO: Your solution to P1 goes here. Other parts should be in separate documents.



%%% WHAT FOLLOWS ARE USEFUL LATEX TOOLS
%%% TODO: PLEASE DELETE WHAT YOU DON'T USE ON EACH HWK, KTHXBYE!



%% To put in blank space (vertical space), use the following command.
%% You can adjust the number of points to suit your tastes.
%% negative points will shift stuff up, if you want
\vspace*{50pt}

%%% Possibly useful stuff
\noindent Possibly useful stuff.

% use the enumerate environment for numbered enumeration (bullets).
% latex will keep track of the numbering for you
% the default, right now, is to number with a), b), c), etc.
\begin{enumerate}
\item enumerate stuff with
\item alphabet characters.
\end{enumerate}

% use the itemize environment for bullets.
% you can either use the default 'bullet', or you can specify your own in [ ]'.
\begin{itemize}
\item[*] or, can control the enumeration,
\item[2.] but then it's up to you to keep track of the numbering system
\item the default is a plain ol' bullet
\end{itemize}

% Regular text is roman font.
Blahblahblah in roman font, indented by default.\\ 

% and \\ ends a line. 
 %Note that a new line is automatically indented 
 %(unless it is the first line in your subsection / question)
 
\noindent \textbf{Blahalala} in bold font, unindented.\\
 
\noindent \emph{blahblahblah} in italics. \\

\noindent \verb+blah blahblah blah+ \texttt{blahblahba} in true-type font.\\ 

\begin{claim}
something happens $\iff$ something else happens.
\end{claim}

\begin{proof}And I'm proving it here. I'll start with the forward direction, and then the backwards.\medskip

\noindent $(\Rightarrow)$ Forward direction proven here.\medskip

\noindent $(\Leftarrow)$ Backwards direction proven here.\end{proof}

\begin{lemma}
Hellloooo lemma!
\end{lemma}

\begin{rt}
As suggested by Arlo...
\end{rt}

%% I can force a new page as follows
\newpage
%and I may want to add a little vertical space
\vspace*{2pt}

Now I will put up a system of equations all lined up n' purdy.
\begin{align*}
LHS &=  n(n-1) \\
&=  2n(n-1)/2 \\
& \stackrel{(*)}{=}  RHS~.
\end{align*}
%% Whatever is directly to the right of the & gets lines up. In this case, the equal signs.

Oh, and this is nifty, although maybe a bit confusing...
\begin{align*}
e_{ij} &=  \left\{ \begin{tabular}{ll} $\{v_i, v_j\}$ & if edges are undirected, \\
						       $(v_i, v_j)$ & otherwise. \end{tabular} \right.
\end{align*}

Here is an environment to write pseudocode:
\begin{algorithmic}
\FOR{$i=1, 2, \ldots, n-1$}
\STATE Store $A[i]+A[i+1]$ in $B[i,i+1]$ 
\FOR{$j=i+2, i+3, \ldots, n$}
\STATE Store $A[j]+B[i,j-1]$ in $B[i,j]$
\ENDFOR
\ENDFOR
\end{algorithmic}

below is some vertical space. You can use this to make room for pictures and stuff that you want to draw by hand.

\vspace{4cm}

See? Vertical space


%% DON'T ERASE THIS LAST LINE
\end{document}
